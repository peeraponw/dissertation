\section{Objectives and scope of the study}\label{ch:intro:se:obj}

The surface influence on the mechanical behavior of materials has been studied for decades. The surface roughness is one of the most important factors that affect the mechanical behavior of materials. In general, the surface roughness effect on the final product can be avoided by later processing. Some may apply electrical discharge machining \autocite{amorimSurfaceModificationTool2017}, some may do coating \autocite{creusCorrosionBehaviourTiN1998}, some may do dipping \autocite{karduckCharacterisationIntermediateLayers1997}.


While it has been discovered that surface condition plays a role in early failure during cold-forming processes, there has not yet been neither investigation nor model development to quantify the effect for further accurately prediction. This dissertation aims to incorporate how to identify, investigate, quantify, and apply numerical model to accurately capture the influence of surface roughness on top of the existing damage model. The objectives of this dissertation are as follows:

\begin{itemize}
    \item  How does surface roughness quantitatively affect the cold formability of steels?
    \item What is the level of agreement between finite element simulations that incorporate surface roughness and actual experimental results?
    \item Can the multiscale modeling approach be generalized to other types of steels and loading conditions?
\end{itemize}

