\section{Objectives and scope of the study}\label{ch:intro:se:obj}

The impact of surface roughness on the mechanical behavior of materials during cold-forming processes has been a subject of scholarly inquiry for several decades \autocite{ahnRelationshipShearStrength2021,bazriMechanicalTribometallurgicalBehavior2022,csertaEffectsGraniteformingMinerals2021,grochePredictionEvolutionSurface2015,heruthunnisaInvestigationsFormabilityTensile2022}. While surface roughness is a critical factor that influences the mechanical properties of the final product, its effects are often mitigated through post-processing techniques such as electrical discharge machining \autocite{amorimSurfaceModificationTool2017}, coating \autocite{creusCorrosionBehaviourTiN1998}, or dipping \autocite{karduckCharacterisationIntermediateLayers1997}. However, these techniques are only applicable after the completion of the cold-forming process. To preemptively address potential failures during the process, numerical simulations employing the Finite Element Method (FEM) in conjunction with damage mechanics models have been developed. Despite these advancements, there remains a paucity of research and model development aimed at quantifying the effects of surface roughness for more accurate predictive modeling.
The primary aim of this dissertation is to develop a comprehensive framework that identifies, investigates, quantifies, and incorporates the influence of surface roughness into existing damage mechanics models. Specifically, the objectives of this study are to answer the following questions:
\begin{itemize}
\item How does surface roughness affect the cold formability/bendability of heavy plates, and how can these effects be quantitatively evaluated using multiscale numerical approaches?
\item How can surface roughness information be incorporated into material models to improve the accuracy of finite element simulations in predicting the damage and fracture behavior of cold-formed materials?
\item How can the influence of hole edge surface condition on hole expansion ratios be quantified using 3D multiscale simulation strategies, and how can these strategies be generalized to handle arbitrary load cases?
\end{itemize}
Furthermore, this research aims to explore the role of surface roughness in the initiation and propagation of cracks during cold-forming processes, as well as the impact of various surface treatments on the overall performance of the formed materials. This will involve a detailed analysis of the microstructural changes that occur during the cold-forming process and their relationship with surface roughness.
Based on the objectives and research questions, the following hypotheses are formulated:
\begin{itemize}
\item Incorporating surface roughness into multiscale models will significantly improve the prediction accuracy for cold formability. This hypothesis posits that the inclusion of surface roughness as a parameter will lead to a better match between simulation and experimental results, particularly in the context of bendability and hole expansion ratios.
\item Surface roughness plays a critical role in determining the mechanical properties and failure behaviors of steels during cold-forming processes. This hypothesis suggests that surface roughness is not merely a secondary factor but a key variable that can influence the ductile damage behavior and fracture mechanisms in steels.
\item The effects of surface roughness can be quantitatively evaluated using the developed multiscale numerical approaches. This hypothesis posits that the models will not only qualitatively describe the influence of surface roughness but also provide numerical metrics that can be validated through experiments.
\end{itemize}
To address these research questions, the dissertation is organized into three main chapters, each focusing on a specific aspect of the influence of surface roughness on cold-forming processes:
\begin{itemize}
\item Chapter \ref{ch:roughness} - a multiscale numerical approach is developed to evaluate the impacts of surface roughness on the cold formability/bendability of heavy plates, using a hybrid damage mechanics model to describe the macroscopic ductile damage behavior of extra abrasion-resistant steel.
\item Chapter \ref{ch:bending} - the damage and fracture behavior of DP1000 steel sheets with different surface treatments are investigated under three-point bending tests, and a multiscale strategy is introduced to quantitatively include surface information into the material model.
\item Chapter \ref{ch:het} - a 3D multiscale simulation strategy is proposed to quantify the influence of hole edge surface condition on hole expansion ratios in DP1000 steel, allowing for the handling of arbitrary load cases and significantly improving the accuracy of simulations compared to state-of-the-art ductile damage mechanics simulations.
\end{itemize}
In conclusion, this dissertation aims to provide a comprehensive understanding of the impact of surface roughness on the mechanical behavior of materials during cold-forming processes. By developing a multiscale modeling framework that incorporates surface roughness information, the research seeks to improve the accuracy of predictive models and contribute to the advancement of cold-forming research. The results of these studies will not only enhance our understanding of the influence of surface roughness on cold-forming processes but also provide valuable insights for the development of more accurate and efficient multiscale modeling approaches. These methodologies have the potential to be applied to other materials and forming processes, further advancing the field of cold-forming research and providing a solid foundation for future work in this area.
