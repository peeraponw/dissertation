\chapter[bending]{Numerical Evaluation of Surface Roughness Influences on Cold Formability of Dual-Phase Steel}\label{ch:bending}
\begin{center}

    \textbf{Peerapon Wechsuwanmanee}, Junhe Lian, Wenqi Liu, Sebastian Münstermann
    
    \vspace{20pt}
    
    Steel Research International 91 (2020) 202000141
    
    \vspace{20pt}
    
    \url{https://doi.org/10.1002/srin.202000141}
    
    \vspace{40pt}
    
\end{center}
% Start chapter
Chapter \ref{ch:bending} investigates the influence of surface roughness on the cold formability of dual-phase steel sheets DP1000, which is a critical factor in automotive industry applications. The study employs a combination of experimental three-point bending tests and numerical simulations to assess the effects of surface roughness on the material's mechanical behavior and fracture characteristics. White light confocal microscopy is utilized to analyze surface topography and gather surface information for smooth and rough surface samples.

The experimental results reveal that different surface treatments result in distinct fracture behaviors, with rougher surfaces exhibiting earlier damage initiation and fracture mechanisms. To address this, a multiscale simulation strategy is introduced, incorporating a hybrid constitutive material model, the MBW model, along with an additional surface factor, ${c_s}$. The surface factor modifies the damage initiation criteria to account for the softened response of the material due to surface condition.

Finite-element (FE) simulations are conducted, which accurately capture the mechanical behavior of smooth surface sheets. To incorporate the surface information into the material model, a submodel is generated for the critical element using data obtained from white light confocal microscopy. Surface localization is considered for calibrating the surface factor, and when applied to every element on the sheet's surface, the calibrated surface factor enables the component-level simulation to accurately capture the early damage and fracture of the rough surface sheet.

The results demonstrate that surface roughness significantly impacts the cold formability of DP1000 steel sheets, with rough surfaces leading to reduced formability compared to smooth surfaces. This highlights the importance of controlling surface roughness to enhance cold formability in dual-phase steel, ultimately contributing to improved performance in automotive applications. 