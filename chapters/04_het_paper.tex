\chapter[het]{3D Multilevel Modeling of Surface Roughness Influences on Hole Expansion Ratios}\label{ch:het}
\begin{center}

    \textbf{Peerapon Wechsuwanmanee}, Junhe Lian, Sebastian Münstermann
    
    \vspace{20pt}
    
    Steel Research International 92 (2021) 202100449
    
    \vspace{20pt}
    
    \url{https://doi.org/10.1002/srin.202100449}
    
    \vspace{40pt}
    
\end{center}
% Start chapter

Chapter \ref{ch:het} present a comprehensive investigation on the effects of surface roughness on hole expansion ratios in advanced high strength steel, specifically focusing on dual-phase steels such as DP1000. To accomplish this, the author employ a novel 3D multiscale simulation strategy that incorporates both microscale and mesoscale strain paths under complex loading conditions, resulting in a more accurate quantification of the impact of surface roughness on material behavior. Finite element analysis (FEA) is utilized to compare experimental results from the Hole Expansion Test (HET) with simulation outcomes under different surface conditions, including perfectly smooth surfaces and calibrated surface factors.

The proposed simulation approach significantly enhances the accuracy of simulations compared to conventional ductile damage mechanics models, addressing existing discrepancies between experimental and simulation results. The study also delves into the microstructure of dual-phase steels and elucidates how it contributes to their remarkable mechanical properties, such as high strength and formability. Furthermore, the importance of considering microstructure in material design is emphasized, and potential solutions for improving mechanical behaviors in dual-phase steels are explored.

In summary, this study provides invaluable insights into the impact of surface roughness on hole expansion ratios in advanced high strength steel and introduces a novel multiscale simulation strategy that significantly improves simulation accuracy. The findings contribute to a more profound understanding of material behavior under complex loading conditions, highlighting the significance of microstructural considerations in material design and offering potential solutions for enhancing mechanical performance in dual-phase steels.