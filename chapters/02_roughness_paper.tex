\chapter[roughness]{Influence of Surface Roughness on Cold Formability in Bending Processes: A Multiscale Modelling Approach with the Hybrid Damage Mechanics Model}\label{ch:roughness}
\begin{center}

    \textbf{Peerapon Wechsuwanmanee}, Junhe Lian, Fuhui Shen, Sebastian Münstermann
    
    \vspace{20pt}
    
    International Journal of Material Forming 14 (2020), 235-248
    
    \vspace{20pt}
    
    \url{https://doi.org/10.1007/s12289-020-01576-7}
    
    \vspace{40pt}
    
\end{center}
% Original
% Chapter \ref{ch:roughness} presents a novel multiscale numerical approach to investigate the influence of surface roughness on the cold formability/bendability of extra abrasion-resistant steel plates in bending processes. The proposed approach integrates a hybrid damage mechanics model to describe the macroscopic ductile damage behavior of the investigated steel, while the surface roughness is characterized using confocal microscopy and statistically represented in a two-dimensional representative volume element (RVE) model.

% The critical ratio between the punch radius and the sample thickness (r/t) is predicted and compared with experimental results obtained from three-point bending tests. The findings reveal that surface roughness has a significant impact on the bendability of steels, with the roughness profile playing a crucial role in cold formability/bendability predictions. The degradation of the mechanical properties due to surface roughness is attributed to severe local strain localization induced by the geometrical inhomogeneity of the surface roughness profile, ultimately leading to early local damage initiation on the surface.

% The multiscale simulation approach demonstrates efficiency in quantitatively describing surface roughness affected ductile damage evolutions, with a significant improvement in the predicted results when considering the surface roughness effects. Consequently, this study emphasizes the importance of accounting for surface roughness in bendability predictions and validates the efficacy of the proposed multiscale modeling scheme for accurately predicting the critical r/t ratio in steel plate bending processes.

% Claude 2
% Chapter \ref{ch:roughness} introduces a novel multiscale numerical approach for investigating the influence of surface roughness on the cold formability/bendability of extra abrasion-resistant steel plates during bending processes. This approach combines a hybrid damage mechanics model to describe the macroscopic ductile damage behavior of the investigated steel and a two-dimensional representative volume element (RVE) model to statistically represent surface roughness, which is characterized using confocal microscopy.

% The critical ratio between the punch radius and the sample thickness (r/t) is predicted and compared with experimental results obtained from three-point bending tests. The results indicate that surface roughness significantly impacts the bendability of steels, with the roughness profile playing a crucial role in cold formability/bendability predictions. The degradation of mechanical properties due to surface roughness is attributed to severe local strain localization induced by the geometrical inhomogeneity of the surface roughness profile, ultimately leading to early local damage initiation on the surface.

% The multiscale simulation approach proves efficient in quantitatively describing surface roughness-affected ductile damage evolutions, with a notable improvement in the predicted results when considering surface roughness effects. This study highlights the importance of accounting for surface roughness in bendability predictions and validates the effectiveness of the proposed multiscale modeling scheme for accurately predicting the critical r/t ratio in steel plate bending processes.

% GPT-4
Chapter \ref{ch:roughness} introduces a novel multiscale numerical method to examine the impact of surface roughness on the cold formability/bendability of extra abrasion-resistant steel plates during bending processes. This approach combines a hybrid damage mechanics model to characterize the macroscopic ductile damage behavior of the investigated steel and a two-dimensional representative volume element (RVE) model to statistically represent surface roughness, which is characterized using confocal microscopy.

The critical ratio between the punch radius and the sample thickness (r/t) is predicted and compared with experimental results from three-point bending tests. The results indicate that surface roughness significantly affects the bendability of steels, with the roughness profile playing a vital role in cold formability/bendability predictions. The degradation of mechanical properties due to surface roughness is attributed to severe local strain localization caused by the geometrical inhomogeneity of the surface roughness profile, leading to early local damage initiation on the surface.

The multiscale simulation approach effectively describes surface roughness influenced ductile damage evolutions, showing a substantial improvement in predicted results when considering surface roughness effects. This study highlights the importance of accounting for surface roughness in bendability predictions and confirms the effectiveness of the proposed multiscale modeling scheme for accurately predicting the critical r/t ratio in steel plate bending processes.