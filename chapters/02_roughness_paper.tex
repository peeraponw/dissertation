\chapter[roughness]{Influence of Surface Roughness on Cold Formability in Bending Processes: A Multiscale Modelling Approach with the Hybrid Damage Mechanics Model}\label{ch:roughness}
\begin{center}

    \textbf{Peerapon Wechsuwanmanee}, Junhe Lian, Fuhui Shen, Sebastian Münstermann
    
    \vspace{20pt}
    
    International Journal of Material Forming 14 (2020), 235-248
    
    \vspace{20pt}
    
    \url{https://doi.org/10.1007/s12289-020-01576-7}
    
    \vspace{40pt}
    
\end{center}
% Start chapter
Chapter II initiates an interest of surface roughness influences on formability in cold-forming processes. Effects on surface roughness was revealed via discrepancy between cold-forming experiments and FE simulations. In this paper, the authors investigate the effect of surface roughness on the cold formability (the ability of a material to be deformed at low temperatures without fracturing) of extra abrasion-resistant steel in bending processes. To do this, they perform both experimental and numerical (computer-based) investigations. They use a novel multiscale numerical approach to quantify the impact of surface roughness on the cold formability of heavy plates. The macroscopic ductile (a type of deformation that occurs in materials under high strain) damage behavior of the steel is described using a hybrid damage mechanics model, which is calibrated using tensile and bending tests. The surface roughness of the steel is characterized using confocal microscopy and incorporated into a two-dimensional computer model. The model is then used to predict the critical ratio between the punch radius and the sample thickness (r/t) in three-point bending tests, and the results are compared with experimental data.

The authors find that surface roughness has a significant effect on the cold formability of the steel, and that their multiscale simulation approach is effective in quantitatively describing the effect of surface roughness on ductile damage evolution. They also find that the roughness of the surface reduces the cold formability of the steel significantly compared to an ideal smooth surface condition. This is due to the severe local strain localization caused by the inhomogeneity of the surface roughness profile, which can lead to early local damage initiation on the surface. In conclusion, the authors recommend considering the effect of surface roughness on cold formability in bending processes, as it can significantly affect the bendability of steels.

Experimental and numerical investigations on XAR450 (eXtra Abrasion Resistant) are provided.

Experiments on XAR450 (eXtra Abrasion Resistant) steel grade to characterize its mechanical behaviors are provided.